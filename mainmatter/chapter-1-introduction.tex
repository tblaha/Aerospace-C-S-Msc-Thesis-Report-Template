\chapter{Introduction}
\label{chapter:intro}

\section{Template description}

This template is made for thesis reports that require a scientific article to be part of the deliverable.
Create a standalone scientific article as a \texttt{PDF} and include it as Chapter 2.

The file \texttt{notation.tex} defines notation macros that are common in your thesis. For example, the expectation operator using \texttt{\textbackslash E}:

\begin{equation}
\E[\pi]{\sum_{t}^{\infty} \x_t}
\end{equation}

The file \texttt{styling.tex} defines official TU Delft colors, and uses primary, secondary and accent definitions to keep the color scheme consistent.
It also defines \texttt{tikz} styles for each shaded and non-shaded blocks for illustration.

Use XeLatex/LuaLatex to compile.

\section{Examples}

Here are a few example citations, for example the PDF: \cite{einstein1905electrodynamics}. Also \cite{mnih2015human}.

A figure with inserted \texttt{PNG} is shown in \cref{fig:example}.
In order to make nice tables, use booktabs and avoid vertical lines, as shown in \cref{tab:example}.
Add illustrations using tikz as shown in \cref{fig:fig_tikz}.

\begin{figure}[ht]
    \centering
    \includegraphics[width=0.4\textwidth]{figures/placeholder_black.png}
    \caption{This is an example figure}
    \label{fig:example}
\end{figure}

\begin{table}[ht]
    \centering
    \caption{Example clean table}
    \label{tab:example}
    \begin{tabular}{@{}ccc@{}}
    \toprule
    Altitude & $1,000 (m)$ & $2,000 (m)$ \\
    \midrule
    P &  0.2  & 0.4  \\ 
    I &  0.01 & 0.02 \\
    D &  0.4  & 0.5  \\
    \bottomrule
    \end{tabular}
\end{table}

\begin{figure}[ht]
    \centering
    \begin{tikzpicture}
        \node[block-primary] (ac) at (0, 0) {Aircraft};
        \node[shaded-primary] (co) at (-6, 0) {Controller};

        \draw[line] (co) -- node [midway, above] {u} (ac);
        \draw[line] (ac) to++ (0, -2) -| node [pos=0.25, below] {x} (co);
        
    \end{tikzpicture}
    \caption{This is an example tikz illustration.}
    \label{fig:fig_tikz}
\end{figure}

\newpage
\section{Research Formulation}
\label{section:intro-reseach-questions}

A few macros are defined to show the research questions and research objective in a box:

\ResearchObjective{The primary objective of this research project is to graduate.}

\ResearchQuestion{1}{What state-of-the-art methods are most suitable for XXX ?}

You can refer to your research question as: RQ \ref{rq:1}, which should generate a hyperlink.

\section{Structure of the Report}
\label{section:intro-report-structure}

% Scope
The purpose of this report is to (...)

% Structure of the paper
The structure of the report is as follows.
Firstly, \cref{chapter:article} shows Einstein's 1905 paper about special relativity as an example inserted standalone article pdf, in \cref{part:article}.
Then,(...)